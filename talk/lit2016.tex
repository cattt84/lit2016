\documentclass[t, utf8x, 10pt]{beamer}
\usepackage[ngerman]{babel}
\usepackage{bera}
\usepackage{listings}
\usepackage{url}

\setbeamertemplate{navigation symbols}{}

\hypersetup{%
  pdftitle={Pythonskripte testen: wie und warum?}
  ,pdfauthor={Gert-Ludwig Ingold <gert.ingold@physik.uni-augsburg.de>}
  ,pdfsubject={Linux Infotag 2016, Augsburg, 16.4.2016}
  ,pdfkeywords={Python, testing, doctests, unittests}
}

\graphicspath{{./img/}}

\definecolor{c1}{hsb}{0.125, 1, 0.8}
\definecolor{c2}{hsb}{0.458, 1, 0.6}
\definecolor{c3}{hsb}{0.791, 1, 0.8}

\lstset{language=Python,
        basicstyle={\ttfamily},
        showstringspaces=false,
        keywordstyle=\color{c2},
        commentstyle=\color{c3},
        stringstyle=\color{c1},
        }


\begin{document}

\begin{frame}[fragile]
 \vspace{1truecm}
 \begin{center}
  \structure{\LARGE Pythonskripte testen: wie und warum?}\\[0.3truecm]
  {\large Gert-Ludwig Ingold}

  \vspace{0.5truecm}
  \begin{minipage}{0.5\textwidth}
   \begin{tiny}
    \begin{lstlisting}[backgroundcolor=\color{black!10}
                       ,language={}]
 **********************************************
 File "example", line 1, in example
 Failed example:
     1+1
 Expected:
     1
 Got:
     2
 **********************************************
 1 items had failures:
     1 of   1 in example
 ***Test Failed*** 1 failures.
    \end{lstlisting}
   \end{tiny}
  \end{minipage}

  \vspace{1truecm}
  \texttt{\normalsize git clone https://github.com/gertingold/lit2016.git}
 \end{center}
\end{frame}


\begin{frame}[fragile]{Was sind Docstrings?}
 \lstinputlisting{../examples/example_1a.py}

 \hrulefill

 \structure{in der Python-Shell:}
 \begin{lstlisting}[language={}]
 >>> help(welcome)

 Help on function welcome in module __main__:

 welcome(name)
     be nice and greet somebody
     name: name of the person
 \end{lstlisting}
\end{frame}


\begin{frame}[fragile]{Ein Gruß an Guido van Rossum}
 \lstinputlisting{../examples/example_1b.py}

 \hrulefill
 \begin{itemize}
  \item die Benutzung der Funktion wird dokumentiert
  \item dabei wird die Syntax der Python-Shell verwendet\\
	\texttt{{>}{>}{>}} \quad Eingabe-Prompt\\
	\texttt{...} \quad Fortsetzungszeile
  \item nach der Ausgabe folgt eine Leerzeile oder der nächste Prompt
  \item \alert{es lässt sich aber auch die korrekte Funktionsweise testen}
 \end{itemize}
\end{frame}


\begin{frame}[fragile]{Ein erster Test}
 Die Funktion sei in der Datei \texttt{example\_1b.py} definiert:

 \begin{lstlisting}[language=bash]
 $ python -m doctest example_1b.py
 $ 
 \end{lstlisting}

 \begin{itemize}
  \item »keine Neuigkeiten sind gute Neuigkeiten«\\
	oder:\\
	erfolgt keine Ausgabe, so wurden alle vorhandenen Tests fehlerfrei
	ausgeführt, sofern sie nicht unterdrückt wurden
  \item es wird das Modul \texttt{doctest} aus der Python-Standardbibliothek
	verwendet\\
	ausführliche Dokumentation unter:
	\url{http://docs.python.org/library/doctest.html}
  \item mit der Option \texttt{-v} wird \texttt{doctest} gesprächiger
 \end{itemize}
\end{frame}


\begin{frame}[fragile]{Und noch einmal mit mehr Details}
	\begin{lstlisting}[language={}]
 $ python -m doctest -v example_1b.py
 Trying:
     welcome('Guido')
 Expecting:
     'Hallo Guido!'
 ok
 1 items had no tests:
     example_1b
 1 items passed all tests:
     1 tests in example_1b.welcome
 1 tests in 2 items.
 1 passed and 0 failed.
 Test passed.
 \end{lstlisting}
\end{frame}


\begin{frame}[fragile]{Ein erster Fehler \dots}
 \lstinputlisting{../examples/example_1c.py}
\end{frame}


\begin{frame}[fragile]{\dots und das Ergebnis}
 \begin{lstlisting}[language={}]
 $ python -m doctest example_1c.py
 **********************************************************************
 File "example_1c.py", line 6, in example_1c.welcome
 Failed example:
     welcome('Guido')
 Expected:
     'Hello Guido!'
 Got:
     'Hallo Guido!'
 **********************************************************************
 1 items had failures:
    1 of   1 in example_1c.welcome
 ***Test Failed*** 1 failures.
 \end{lstlisting}

 \begin{itemize}
  \item bei Fehlern werden Details auch ohne die Option \texttt{-v} ausgegeben
 \end{itemize}
\end{frame}


\begin{frame}[c]{Erst die Tests, dann das Programmieren}
 \begin{center}
  \begin{Large}
   \setlength\fboxsep{0.4truecm}	  
   \fbox{\parbox{0.85\textwidth}{\structure{Test-driven development (TDD):}\\
                                 Formuliere erst die Tests und entwickle dann das
                                 Programm bis alle Tests fehlerfrei ausgeführt werden.}
	}
  \end{Large}
 \end{center}
\end{frame}


\begin{frame}[fragile]{Wunschliste als Tests}
 \lstinputlisting{../examples/example_1d.py}
\end{frame}


\begin{frame}[fragile]{Fehler, die es zu beseitigen gilt}
 \begin{tiny}
	 \begin{lstlisting}[language={}]
 **********************************************************************
 File "example_1d.py", line 6, in example_1d.welcome
 Failed example:
     welcome()
 Exception raised:
     Traceback (most recent call last):
       File "python3.5/doctest.py", line 1320, in __run
	 compileflags, 1), test.globs)
       File "<doctest example_1d.welcome[0]>", line 1, in <module>
         welcome()
     TypeError: welcome() missing 1 required positional argument: 'name'
 **********************************************************************
 File "example_1d.py", line 9, in example_1d.welcome
 Failed example:
     welcome(lang='de')
 Exception raised:
     Traceback (most recent call last):
       File "python3.5/doctest.py", line 1320, in __run
         compileflags, 1), test.globs)
       File "<doctest example_1d.welcome[1]>", line 1, in <module>
         welcome(lang='de')
     TypeError: welcome() got an unexpected keyword argument 'lang'
 **********************************************************************
 File "example_1d.py", line 12, in example_1d.welcome
 Failed example:
     welcome('Guido')
 Expected:
     'Hello Guido!'
 Got:
     'Hallo Guido!'
 **********************************************************************
 1 items had failures:
    3 of   3 in example_1d.welcome
 ***Test Failed*** 3 failures.
  \end{lstlisting}
 \end{tiny}
\end{frame}


\begin{frame}[fragile]{Ausnahmen sind manchmal gewollt}
 \begin{tiny}
  \lstinputlisting{../examples/example_1e.py}
 \end{tiny}

 \begin{itemize}
  \item eine nicht implementierte Sprache soll zu einem \texttt{ValueError} führen
 \end{itemize}
\end{frame}


\begin{frame}[fragile]{Behandlung von Ausnahmen}
 \begin{itemize}
  \item Problem: die Ausgabe ist häufig komplex
 \end{itemize}
 \begin{footnotesize}
  \begin{lstlisting}[language={}]
 Traceback (most recent call last):
   File "example_1g.py", line 25, in welcome
     greeting = greetings[lang]
 KeyError: 'nl'

 During handling of the above exception, another exception occurred:

 Traceback (most recent call last):
   File "example_1g.py", line 33, in <module>
     welcome('Guido', 'nl')
   File "example_1g.py", line 28, in welcome
     raise ValueError(errmsg)
 ValueError: unknown language: nl
  \end{lstlisting}
 \end{footnotesize}
 \begin{itemize}
  \item es genügt aber:
 \end{itemize}
 \begin{footnotesize}
  \begin{lstlisting}
 """
 >>> welcome('Guido', 'nl')
 Traceback (most recent call last):
 ValueError: unknown language: nl
 """
  \end{lstlisting}
 \end{footnotesize}

\end{frame}


\begin{frame}[fragile]{Direktiven für \texttt{doctest}}
 \begin{itemize}
  \item Platzhalter für beliebige Ausgabe
 \end{itemize}
 \begin{footnotesize}
  \begin{lstlisting}
 """
 >>> welcome('Guido', 'nl')  # doctest: +ELLIPSIS
 Traceback (most recent call last):
 ValueError: ...
 """
  \end{lstlisting}
 \end{footnotesize}

 \begin{itemize}
  \item der Test soll vorläufig nicht durchgeführt werden
 \end{itemize}
 \begin{footnotesize}
  \begin{lstlisting}
 """
 >>> welcome('Guido', 'nl')  # doctest: +SKIP
 'Goed dag Guido!'
 """
  \end{lstlisting}
 \end{footnotesize}

 \begin{itemize}
  \item siehe die Dokumentation für weitere Direktiven:
	\url{http://docs.python.org/library/doctest.html}
  \item interessant ist z.B. noch \texttt{+NORMALIZE\_WHITESPACE}
 \end{itemize}
\end{frame}

\end{document}
