\documentclass[t, utf8x, 10pt]{beamer}
\usepackage[ngerman]{babel}
\usepackage{bera}
\usepackage{fontawesome}
\usepackage{listings}
\usepackage{url}

\setbeamertemplate{navigation symbols}{}

\hypersetup{%
  pdftitle={Pythonskripte testen: wie und warum?}
  ,pdfauthor={Gert-Ludwig Ingold <gert.ingold@physik.uni-augsburg.de>}
  ,pdfsubject={Linux Infotag 2016, Augsburg, 16.4.2016}
  ,pdfkeywords={Python, testing, doctests, unittests}
}

\graphicspath{{./img/}}

\definecolor{c1}{hsb}{0.125, 1, 0.8}
\definecolor{c2}{hsb}{0.458, 1, 0.6}
\definecolor{c3}{hsb}{0.791, 1, 0.8}
\definecolor{pro}{rgb}{0, 0.6, 0}
\definecolor{contra}{rgb}{0.8, 0, 0}

\lstset{language=Python,
        basicstyle={\ttfamily},
        showstringspaces=false,
        keywordstyle=\color{c2},
        commentstyle=\color{c3},
        stringstyle=\color{c1}
        }

\newcommand\pro{\textcolor{pro}{\faicon{smile-o}}}
\newcommand\contra{\textcolor{contra}{\faicon{frown-o}}}


\begin{document}

\begin{frame}[fragile]
 \vspace{1truecm}
 \begin{center}
  \structure{\LARGE Pythonskripte testen: wie und warum?}\\[0.3truecm]
  {\large Gert-Ludwig Ingold}

  \vspace{0.5truecm}
  \begin{minipage}{0.5\textwidth}
   \begin{tiny}
    \begin{lstlisting}[backgroundcolor=\color{black!10}
                       ,language={}]
 **********************************************
 File "example", line 1, in example
 Failed example:
     1+1
 Expected:
     1
 Got:
     2
 **********************************************
 1 items had failures:
     1 of   1 in example
 ***Test Failed*** 1 failures.
    \end{lstlisting}
   \end{tiny}
  \end{minipage}

  \vspace{1truecm}
  \faicon{github}
  \texttt{\normalsize git clone https://github.com/gertingold/lit2016.git}
 \end{center}
\end{frame}


\begin{frame}[c]{Wer testet seine Programme?}
 \begin{itemize}
  \item[\pro]    Programme werden praktisch immer bei der Programmentwicklung
	         durch Vergleich mit dem erwarteten Verhalten getestet.
  \item[\contra] Das passiert oft sehr informell und kaum reproduzierbar.
 \end{itemize}
\end{frame}


\begin{frame}{Einige Überlegungen zum Testen}
\end{frame}


\begin{frame}[fragile]{Was sind Docstrings?}
 \lstinputlisting{../examples/doctests/example1_v1.py}

 \hrulefill

 \structure{in der Python-Shell:}
 \begin{lstlisting}[language={}]
 >>> help(welcome)

 Help on function welcome in module __main__:

 welcome(name)
     be nice and greet somebody
     name: name of the person
 \end{lstlisting}
\end{frame}


\begin{frame}{Ein Gruß an Guido van Rossum}
 \lstinputlisting{../examples/doctests/example1_v2.py}

 \hrulefill
 \begin{itemize}
  \item die Benutzung der Funktion wird dokumentiert
  \item dabei wird die Syntax der Python-Shell verwendet\\
	\texttt{{>}{>}{>}} \quad Eingabe-Prompt\\
	\texttt{...} \quad Fortsetzungszeile
  \item nach der Ausgabe folgt eine Leerzeile oder der nächste Prompt
  \item \alert{es lässt sich aber auch die korrekte Funktionsweise testen}
 \end{itemize}
\end{frame}


\begin{frame}[fragile]{Ein erster Test}
 Die Funktion sei in der Datei \texttt{example1\_v2.py} definiert:

 \begin{lstlisting}[language=bash]
 $ python -m doctest example1_v2.py
 $ 
 \end{lstlisting}

 \begin{itemize}
  \item »keine Neuigkeiten sind gute Neuigkeiten«\\
	oder:\\
	erfolgt keine Ausgabe, so wurden alle vorhandenen Tests fehlerfrei
	ausgeführt, sofern sie nicht unterdrückt wurden
  \item es wird das Modul \texttt{doctest} aus der Python-Standardbibliothek
	verwendet\\
	ausführliche Dokumentation unter:
	\url{http://docs.python.org/library/doctest.html}
  \item mit der Option \texttt{-v} wird \texttt{doctest} gesprächiger
 \end{itemize}
\end{frame}


\begin{frame}[fragile]{Und noch einmal mit mehr Details}
	\begin{lstlisting}[language={}]
 $ python -m doctest -v example1_v2.py
 Trying:
     welcome('Guido')
 Expecting:
     'Hallo Guido!'
 ok
 1 items had no tests:
     example1_v2
 1 items passed all tests:
     1 tests in example1_v2.welcome
 1 tests in 2 items.
 1 passed and 0 failed.
 Test passed.
 \end{lstlisting}
\end{frame}


\begin{frame}{Ein erster Fehler \dots}
 \lstinputlisting{../examples/doctests/example1_v3.py}
\end{frame}


\begin{frame}[fragile]{\dots und das Ergebnis}
 \begin{lstlisting}[language={}]
 $ python -m doctest example1_v3.py
 **********************************************************************
 File "example1_v3.py", line 6, in example1_v3.welcome
 Failed example:
     welcome('Guido')
 Expected:
     'Hello Guido!'
 Got:
     'Hallo Guido!'
 **********************************************************************
 1 items had failures:
    1 of   1 in example1_v3.welcome
 ***Test Failed*** 1 failures.
 \end{lstlisting}

 \begin{itemize}
  \item bei Fehlern werden Details auch ohne die Option \texttt{-v} ausgegeben
 \end{itemize}
\end{frame}


\begin{frame}[c]{Erst die Tests, dann das Programmieren}
 \begin{center}
  \begin{Large}
   \setlength\fboxsep{0.4truecm}	  
   \fbox{\parbox{0.85\textwidth}{\structure{Test-driven development (TDD):}\\
                                 Formuliere erst die Tests und entwickle dann das
                                 Programm bis alle Tests fehlerfrei ausgeführt werden.}
	}
  \end{Large}
 \end{center}
\end{frame}


\begin{frame}{Wunschliste als Tests}
 \lstinputlisting{../examples/doctests/example1_v4.py}
\end{frame}


\begin{frame}[fragile]{Fehler, die es zu beseitigen gilt}
 \begin{tiny}
	 \begin{lstlisting}[language={}]
 **********************************************************************
 File "example1_v4.py", line 6, in example1_v4.welcome
 Failed example:
     welcome()
 Exception raised:
     Traceback (most recent call last):
       File "python3.5/doctest.py", line 1320, in __run
	 compileflags, 1), test.globs)
       File "<doctest example1_v4.welcome[0]>", line 1, in <module>
         welcome()
     TypeError: welcome() missing 1 required positional argument: 'name'
 **********************************************************************
 File "example1_v4.py", line 9, in example1_v4.welcome
 Failed example:
     welcome(lang='de')
 Exception raised:
     Traceback (most recent call last):
       File "python3.5/doctest.py", line 1320, in __run
         compileflags, 1), test.globs)
       File "<doctest example1_v4.welcome[1]>", line 1, in <module>
         welcome(lang='de')
     TypeError: welcome() got an unexpected keyword argument 'lang'
 **********************************************************************
 File "example1_v4.py", line 12, in example1_v4.welcome
 Failed example:
     welcome('Guido')
 Expected:
     'Hello Guido!'
 Got:
     'Hallo Guido!'
 **********************************************************************
 1 items had failures:
    3 of   3 in example1_v4.welcome
 ***Test Failed*** 3 failures.
  \end{lstlisting}
 \end{tiny}
\end{frame}


\begin{frame}{Ausnahmen sind manchmal gewollt}
 \begin{tiny}
  \lstinputlisting{../examples/doctests/example1_v5.py}
 \end{tiny}

 \begin{itemize}
  \item eine nicht implementierte Sprache soll zu einem \texttt{ValueError} führen
 \end{itemize}
\end{frame}


\begin{frame}[fragile]{Behandlung von Ausnahmen}
 \begin{itemize}
  \item Problem: die Ausgabe ist häufig komplex
 \end{itemize}
 \begin{footnotesize}
  \begin{lstlisting}[language={}]
 Traceback (most recent call last):
   File "example1_v6.py", line 25, in welcome
     greeting = greetings[lang]
 KeyError: 'nl'

 During handling of the above exception, another exception occurred:

 Traceback (most recent call last):
   File "example1_v6.py", line 33, in <module>
     welcome('Guido', 'nl')
   File "example1_v6.py", line 28, in welcome
     raise ValueError(errmsg)
 ValueError: unknown language: nl
  \end{lstlisting}
 \end{footnotesize}
 \begin{itemize}
  \item als Doctest genügt aber:
 \end{itemize}
 \begin{footnotesize}
 \lstinputlisting[linerange=16-18]{../examples/doctests/example1_v6.py}
 \end{footnotesize}
\end{frame}


\begin{frame}[fragile]{Direktiven für \texttt{doctest}}
 \begin{itemize}
  \item Platzhalter für beliebige Ausgabe
 \end{itemize}
 \begin{footnotesize}
  \lstinputlisting[linerange=16-18]{../examples/doctests/example1_v7.py}
 \end{footnotesize}

 \begin{itemize}
  \item der Test soll vorläufig nicht durchgeführt werden
 \end{itemize}
 \begin{footnotesize}
  \lstinputlisting[linerange=20-21]{../examples/doctests/example1_v7.py}
 \end{footnotesize}

 \begin{itemize}
  \item siehe die Dokumentation für weitere Direktiven:
	\url{http://docs.python.org/library/doctest.html}
  \item interessant ist z.B. noch \texttt{+NORMALIZE\_WHITESPACE}
 \end{itemize}
\end{frame}


\begin{frame}[fragile]{Doctests in beliebigem Text}
 Doctests sind nicht auf die Verwendung Docstrings begrenzt, sondern können in
 beliebige Textdateien eingebettet und dort getestet werden.

 \begin{columns}
  \begin{column}{0.5\textwidth}
   \texttt{example2.txt:}
   \begin{footnotesize}
    \lstinputlisting[language={}]{../examples/doctests/example2.txt}
   \end{footnotesize}
  \end{column}
  \begin{column}{0.5\textwidth}
   \begin{footnotesize}
    \begin{lstlisting}[language={}]
 $ python -m doctest -v example2.txt
 Trying:
     x = 1
 Expecting nothing
 ok
 Trying:
     if x < 0:
         print('x ist negativ')
     else:
         print('x ist nicht negativ')
 Expecting:
     x ist nicht negativ
 ok
 1 items passed all tests:
    2 tests in example2.txt
 2 tests in 1 items.
 2 passed and 0 failed.
 Test passed.
    \end{lstlisting}
   \end{footnotesize}
  \end{column}
 \end{columns}
\end{frame}


\begin{frame}[c]{Vor- und Nachteile von Doctests}
 \begin{itemize}
  \item[\pro]    leicht zu schreiben
  \item[\pro]    unterstützt die Dokumentation durch Beschreibung des
	         Benutzerinterfaces
  \item[\pro]    lässt sich -- im Gegensatz zum Rest der Dokumentation -- leicht
	         auf Korrektheit überprüfen
  \item[\pro]    kann zum Testen von Code in jeder Art von Text verwendet
                 werden\\[0.5truecm]
  \item[\contra] nicht gut für aufwändigere Testsuiten geeignet, zumindest nicht
	         in Docstrings
  \item[\contra] nicht für alle Testszenarien geeignet, z.B. Tests
	         von numerischen Codes bei denen das Ergebnis typischerweise
		 nur näherungsweise korrekt ist
 \end{itemize}
\end{frame}


\begin{frame}[fragile]{Pascalsches Dreieck}
 \begin{small}
  \lstinputlisting{../examples/unittests/pascal_v1.py}

  \begin{lstlisting}[language={}]
 $ python pascal_v1.py 
 0             1           
 1           1  1          
 2          1  2  1        
 3        1  3  3  1       
 4       1  4  6  4  1     
 5     1  5 10 10  5  1    
 6    1  6 15 20 15  6  1 
  \end{lstlisting}
 \end{small}
\end{frame}


\begin{frame}[fragile]{Erste Unittests}
 \texttt{test\_pascal\_v1.py:}
 \begin{scriptsize}
  \lstinputlisting{../examples/unittests/test_pascal_v1.py}

  \hrulefill

  \begin{lstlisting}[language={}]
 $ python test_pascal_v1.py 
 ...
 ----------------------------------------------------------------------
 Ran 3 tests in 0.000s
 
 OK
  \end{lstlisting}
 \end{scriptsize}
 \begin{itemize}
	 \item die Namen von Test-Methoden müssen mit \texttt{test} beginnen
 \end{itemize}
\end{frame}


\begin{frame}{Ein Fehler ...}
 \texttt{test\_pascal\_v2.py:}
 \begin{small}
  \lstinputlisting{../examples/unittests/test_pascal_v2.py}
 \end{small}
 \begin{itemize}
  \item Der Fehler wurde hier in den Test eingebaut, er könnte aber genauso
        gut im Programm stecken.
 \end{itemize}
\end{frame}


\begin{frame}[fragile]{... und das Ergebnis}
 \begin{scriptsize}
  \begin{lstlisting}[language={}]
 $ python test_pascal_v2.py
 ..F
 ======================================================================
 FAIL: test_n5 (__main__.TestExplicit)
 ----------------------------------------------------------------------
 Traceback (most recent call last):
   File "test_pascal_v2.py", line 12, in test_n5
     self.assertEqual(list(pascal(5)), [1, 4, 6, 4, 1])
 AssertionError: Lists differ: [1, 5, 10, 10, 5, 1] != [1, 4, 6, 4, 1]

 First differing element 1:
 5
 4

 First list contains 1 additional elements.
 First extra element 5:
 1

 - [1, 5, 10, 10, 5, 1]
 + [1, 4, 6, 4, 1]

 ----------------------------------------------------------------------
 Ran 3 tests in 0.001s

 FAILED (failures=1)
  \end{lstlisting}
 \end{scriptsize}
\end{frame}


\begin{frame}[fragile]{Ein erwarteter Fehler}
 Auszug aus \texttt{test\_pascal\_v3.py:}
 \begin{small}
 \lstinputlisting[linerange=1-1]{../examples/unittests/test_pascal_v3.py}
 [...]
 \lstinputlisting[linerange=4-4]{../examples/unittests/test_pascal_v3.py}
 [...]
 \lstinputlisting[linerange=11-13]{../examples/unittests/test_pascal_v3.py}

 \hrulefill

  \begin{lstlisting}[language={}]
 $ python test_pascal_v3.py
 ..x
 ------------------------------------------------------------------
 Ran 3 tests in 0.001s

 OK (expected failures=1)
  \end{lstlisting}
 \end{small}
\end{frame}


\begin{frame}{Was tun bei größeren Argumenten?}
bei großen Argumenten ist es nicht mehr sinnvoll, gegen das explizite Resultat
zu testen $\rightarrow$ andersartige Tests sind erforderlich

\vspace{\baselineskip}
binomische Formeln:
\begin{displaymath}
 \begin{aligned}
  (a+b)^2 &= 1\cdot a^2+2\cdot ab+1\cdot b^2\\
  (a-b)^2 &= 1\cdot a^2-2\cdot ab+1\cdot b^2
 \end{aligned}
\end{displaymath}
Die Koeffizienten sind die Einträge bzw. die Einträge mit alternierendem Vorzeichen
aus dem pascalschen Dreieck für $n=2$.

\vspace{\baselineskip}
Für $a=b=1$:
\begin{displaymath}
 \begin{aligned}
  1+2+1 &= 2^2\hspace{0.2truecm} \text{allgemein:}\ 2^n \\
  1-2+1 &= 0\hspace{0.35truecm}    \text{gilt für alle $n$} \\
 \end{aligned}
\end{displaymath}

\vspace{1\baselineskip}
weitere Möglichkeit:

Überprüfe, ob sich aufeinanderfolgende Zeilen des pascalschen Dreiecks durch
Addition von benachbarten Elementen erzeugen lassen.
\end{frame}


\begin{frame}{Implementation der Tests (I)}
 \begin{small}
  [...]
  \lstinputlisting[linerange=15-22]{../examples/unittests/test_pascal_v4.py}
  [...]
  \lstinputlisting[linerange=33-37]{../examples/unittests/test_pascal_v4.py}
  [...]
 \end{small}
\end{frame}


\begin{frame}{Implementation der Tests (II)}
 \begin{small}
  \lstinputlisting[linerange=1-1]{../examples/unittests/test_pascal_v4.py}
  [...]
  \lstinputlisting[linerange=24-31]{../examples/unittests/test_pascal_v4.py}
  [...]
  \lstinputlisting[linerange=38-40]{../examples/unittests/test_pascal_v4.py}
  [...]
 \end{small}
\end{frame}


\begin{frame}{Testen auf Ausnahmen}
 Auszug aus \texttt{pascal\_v2.py}:
 \begin{footnotesize}
  \lstinputlisting[linerange=1-5]{../examples/unittests/pascal_v2.py}
 \end{footnotesize}
 [...]

 \begin{itemize}
  \item Allerdings wird mit der Ausführung des Codes erst begonnen, wenn
	ein Rückgabewert benötigt wird. Dazu definieren wir uns eine
	Funktion \texttt{pascal\_1st\_elem}.
 \end{itemize}

 \begin{footnotesize}
  \lstinputlisting[linerange=33-36]{../examples/unittests/test_pascal_v5.py}
 \end{footnotesize}

 Argumente von \texttt{assertRaises}:
 \begin{enumerate}
  \item Ausnahme
  \item ausführbares Objekt (Funktion, Methode)
  \item Argumente, die an das ausführbare Objekt übergeben werden
 \end{enumerate}
\end{frame}


\begin{frame}{Erweiterung auf Floats}
 Unsere \texttt{pascal}-Methode kann leicht auf Float-Argumente angepasst werden.
 
 Beispiel:
 \begin{displaymath}
  \sqrt[3]{1+x} = 1+\frac{1}{3}x-\frac{1}{9}x^2+\frac{5}{81}x^3+\ldots
 \end{displaymath}

 \texttt{pascal\_v3.py:}
 \lstinputlisting{../examples/unittests/pascal_v3.py}

 Wenn $n$ keine nichtnegative ganze Zahl ist, gibt der Generator potentiell unendlich
 viele Werte zurück.
\end{frame}


\begin{frame}{Testen mit Floats}
 Auszug aus \texttt{test\_pascal\_v6.py}:
 \begin{small}
  \lstinputlisting[linerange=2-3]{../examples/unittests/test_pascal_v6.py}
  [...]
  \lstinputlisting[linerange=15-20]{../examples/unittests/test_pascal_v6.py}
  [...]
  \lstinputlisting[linerange=43-47]{../examples/unittests/test_pascal_v6.py}
 \end{small}
 \begin{itemize}
  \item unpassende Tests können mit dem \texttt{skip}-Dekorator deaktiviert werden
 \end{itemize}
\end{frame}


\begin{frame}[fragile]{Achtung, Rundungsfehler!}
 \begin{scriptsize}
  \begin{lstlisting}[language={}]
 $ python test_pascal_v6.py
 s...Fsss
 ======================================================================
 FAIL: test_one_third (__main__.TestFractional)
 ----------------------------------------------------------------------
 Traceback (most recent call last):
   File "test_pascal_v6.py", line 20, in test_one_third
     self.assertEqual(result, expected)
 AssertionError: Lists differ: [1, 0.3333333333333333, -0.11111111111111112,
 0.0617283950617284] != [1, 0.3333333333333333, -0.1111111111111111,
 0.06172839506172839]

 First differing element 2:
 -0.11111111111111112
 -0.1111111111111111

 - [1, 0.3333333333333333, -0.11111111111111112, 0.0617283950617284]
 ?                                            -                   ^

 + [1, 0.3333333333333333, -0.1111111111111111, 0.06172839506172839]
 ?                                                               ^^


 ----------------------------------------------------------------------
 Ran 8 tests in 0.001s

 FAILED (failures=1, skipped=4)
  \end{lstlisting}
 \end{scriptsize}
\end{frame}


\begin{frame}{Toleranz bei Floats}
 \begin{itemize}
  \item Verwendung von \texttt{math.isclose} (ab Python 3.5)
  \item \texttt{self.assertAlmostEqual}
  \item für Listen und NumPy-Arrays: \texttt{numpy.testing.assert\_allclose}\\
	damit lässt sich auch die Toleranz gut einstellen
 \end{itemize}

 Auszug aus \texttt{test\_pascal\_v7.py:}
 \lstinputlisting[linerange=3-3]{../examples/unittests/test_pascal_v7.py}
 [...]
 \lstinputlisting[linerange=16-21]{../examples/unittests/test_pascal_v7.py}

\end{frame}

\end{document}
